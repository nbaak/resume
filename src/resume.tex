\documentclass{scrartcl}
\usepackage[ngerman]{babel}
\usepackage{geometry}
\geometry{a4paper, margin=1in}
\usepackage{enumitem}
\usepackage{graphicx}  % For including images
% \usepackage{parskip}   % For better spacing
\usepackage{xifthen}   % For conditional statements

% Define default values for personal data
\newcommand{\name}{Max Mustermann}
\newcommand{\birthdate}{1. Januar 1990}
\newcommand{\address}{Musterstraße 123, 12345 Musterstadt}
\newcommand{\phone}{+49 123 456789}
\newcommand{\email}{max.mustermann@example.com}
\newcommand{\photo}{photo.jpg} % Default photo

% Optional field (not defined by default)
% \newcommand{\github}{https://github.com/mustermann}

% Define the \personaldata command with photo and optional GitHub
\newcommand{\personaldata}{
    \begin{minipage}{0.75\textwidth}
        \begin{tabular}{rl}
        Name: & \name \\
        Geburtsdatum: & \birthdate \\
        Adresse: & \address \\
        Telefon: & \phone \\
        E-Mail: & \email \\
        \ifthenelse{\isundefined{\github}}{}{GitHub: & \github \\}
        \end{tabular}
    \end{minipage}
    \begin{minipage}{0.24\textwidth}
        \raggedleft
        \includegraphics[width=\linewidth]{\photo} % Include the photo
    \end{minipage}
    \vspace{.2cm} % Add some space after personal data
}

% Define the \educationitem command (yearinfo, school, degree)
\newcommand{\educationitem}[3]{%
    \item #1 \textbf{#2} \textit{#3}
}

% Define the \education command
\newcommand{\education}[1]{
    \section*{Bildung}
    \begin{itemize}[left=0pt, label={--}]
        #1
    \end{itemize}
}

% Define the \careeritem command (yearinfo, position, company)
\newcommand{\careeritem}[3]{%
    \item #1 \hfill \textbf{#2} \\
    \textit{#3}
}

% Define the \career command
\newcommand{\career}[1]{
    \section*{Berufserfahrung}
    \begin{itemize}[left=0pt, label={--}]
        #1
    \end{itemize}
}

% Define the \skills command (list of skills)
\newcommand{\skills}[1]{
    \section*{Fähigkeiten}
    \begin{itemize}[left=0pt, label={--}]
        #1
    \end{itemize}
}

\begin{document}

% Overwrite the default personal data
\renewcommand{\name}{Eduard Example}
\renewcommand{\birthdate}{15. Februar 1985}
\renewcommand{\address}{Beispielweg 456, 98765 Beispielstadt}
\renewcommand{\phone}{+49 987 654321}
\renewcommand{\email}{eduard.example@example.com}
\renewcommand{\photo}{photo.jpg}
\newcommand{\github}{https://github.com/eduardexample}  % Optional GitHub link

% Personal Data Section at the top
\personaldata

% Education Section with custom entries
\education{
    \educationitem{Okt 2012 -- Sep 2014}{Beispiel Universität, Musterstadt}{M.Sc. Informatik}
    \educationitem{Okt 2009 -- Sep 2012}{Beispiel Universität, Musterstadt}{B.Sc. Informatik}
}

% Career Section with custom entries
\career{
    \careeritem{Jan 2020 -- Present}{Senior Software Engineer}{Tech Company, Musterstadt}
    \careeritem{Jan 2017 -- Dec 2019}{Software Engineer}{Another Tech Company, Beispielstadt}
    \careeritem{Jul 2014 -- Dec 2016}{Junior Developer}{Some Tech Firm, Großstadt}
}

% Skills Section with custom entries
\skills{
    \item Programmierung in Python
    \item Programmierung in C\#
    \item Programmierung in Php
    \item Programmierung in C
    \item Webentwicklung mit HTML, CSS, JavaScript
    \item Datenbankmanagement mit SQL (Oracle, Maria, MySQL)
    \item Nutzung von Versionierungssystemen wie Git
}

\end{document}
